\documentclass{article}
\usepackage[margin=1.5in]{geometry}

\usepackage{booktabs}
\usepackage{hyperref}
\usepackage{pgf}
\usepackage[round,authoryear]{natbib}

% Function to get environment variables for conditional compilation
% of abridged (conference) and extended (arXiv) versions.
\usepackage{etoolbox}
\usepackage{xstring}
\usepackage{catchfile}
\def\newtemp{}%
\newcommand{\getenv}[2][]{%
    \CatchFileEdef{\temp}{"|kpsewhich --var-value #2"}{}%
    \StrGobbleRight{\temp}{1}[\newtemp]%  Delete the trailing whitespace character
    \if\relax\detokenize{#1}\relax\temp\else\edef#1{\newtemp}\fi%
}%

% Convenience wrapper around our environment variable.
% Can be called with one argument for extended-only blocks.
\getenv[\ABRIDGED]{ABRIDGED}
\newcommand\ifextended[2]{%
    \ifdefstring{\ABRIDGED}{true}{%
        #2%
    }{%
        #1%
    }%
}

% Necessary for using Matplotlib's .pgf output.
\usepackage[utf8]{inputenc}
\DeclareUnicodeCharacter{2212}{-}
\renewcommand{\sffamily}{}


\title{Reproducible Papers with\\ \LaTeX, Python, and Makefiles}
\author{James A. Preiss}
\date{\today}


\begin{document}

\maketitle

\begin{abstract}
    We present a framework for completely reproducible academic papers.%
    \ifextended{}{\footnote{%
        An extended version of this paper is available at \url{arxiv.org/abs/9901.0001}.%
    }}
\end{abstract}

\section{Introduction}

The Unix program \texttt{make} was mainly inspired by transitive dependencies in C program compilation \citep{feldman1979make},
but it is useful for other processes too.


\section{Generate \LaTeX\ and figures from scratch}

\input{quadratic_gen.tex}

\begin{figure}[h]
    \centering
    \input{figures/quadratic_roots.pgf}
    \caption{A quadratic polynomial with two distinct real roots.}
\end{figure}


\section{Generate \LaTeX\ and figures from data files}

\begin{table}[h]
    \centering
    \input{sine_taylor_gen.tex}
    \caption{%
        Maximum absolute error of Taylor series for $\sin x$
        on interval $x \in [-\pi, \pi]$.
    }
\end{table}

\begin{figure}[h]
    \centering
    \input{figures/sine_taylor.pgf}
    \caption{Taylor series approximations of sine function.}
\end{figure}


\bibliographystyle{plainnat}
\bibliography{reproducible}

\ifextended{
    \newpage
    \appendix
    \section{Appendix}
    This Appendix should only appear in the unabridged version.
}

\end{document}
